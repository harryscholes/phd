% !TEX root = main.tex

\chapter{Conclusions and future directions}
\label{chapter:discussion}

The overarching theme of this thesis was the development and application of protein function prediction methods.
\begin{itemize}
    \item In \ref{chapter:fly}, we identified proteins whose expression is significantly altered in an Alzheimer's disease model, focussing on the functional consequences of proteome dysregulation.
    \item In \ref{chapter:metagenomes}, we identified putative plastic hydrolase enzymes in metagenomes.
    \item In \ref{chapter:network-fusion}, we developed a feature learning method that generates embeddings of proteins according to their multi-network context.
    \item In \ref{chapter:yeast}, we predicted fission yeast protein functions using network embeddings, evolutionary information from CATH-FunFams and fitness data from phenomic screens of gene deletion mutants.
\end{itemize}
In this chapter, we identify commonalities between these research projects, draw general conclusions from them, and sketch out future directions of research in these areas.

\section{Protein function prediction methods that are not restricted to a single species}

The work in this thesis made liberal use of high-quality protein network data for protein function prediction (\ref{chapter:fly,chapter:network-fusion,chapter:yeast}). These studies focussed on two model organisms---\emph{Schizosaccharomyces pombe} and \emph{Drosophila melanogaster}---that have been the subject of intense research for many decades. Due to the aggregation of information across a large number of orthogonal experiments, we can have a reasonable degree of confidence in network data from such model organisms.

Whilst network-based methods predict protein function well, they do have their drawbacks. Firstly, they are restricted to organisms that have network data---let alone high-quality data from well characterised species---and are often constrained to be applied to a single organism. Secondly, network-based methods are not applicable to novel data, such as the metagenomic protein sequences we encountered in \ref{chapter:metagenomes}.
Finally, network data can be noisy, as many databases infer edges between proteins from correlations in gene expression, such as from RNA-Seq experiments.
Physically-interacting proteins in humans, mice and budding yeast only have a slightly higher correlation in their gene expression than randomly selected pairs of proteins \cite{Bhardwaj2005}.
Despite this, network data has been improving and will continue to do so as high-throughput experiments become more reliable and interactions are confirmed by independent studies.

Ideally, protein function prediction methods would be species-agonostic, such that they are able to use protein information from many different species. One desirable goal is to use \emph{all} protein sequence information. CATH approximates this goal by learning patterns directly from protein structures and sequences, disregarding any associated metadata. The resulting protein family HMMs can be applied to any arbitrary protein sequence to assign the sequence to a family, followed by any functions associated with the family's sequences. We used this method successfully in \ref{chapter:yeast} to predict GO term annotations in CAFA 4, preliminarily achieving first place for molecular function terms. Sequence embedding methods also achieve this goal by embedding arbitrary length sequences in a fixed dimensional space (\ref{chapter:introduction}). Off-the-shelf supervised machine learning models can be trained on these embeddings to perform function prediction. Such models can be applied to large, diverse protein sequence data sets because all of the sequence information can be integrated via the sequence embedding.

\section{Determine which types of data and models are most predictive of protein function}

Protein function prediction performance is limited by the data used to train predictive models. As the quality, volume and coverage of training data is increased, one would expect model performance to increase. At some point, this relationship may break down as higher-order effects, that are not present in the training data, cannot be accounted for. It is reasonable to assume, however, that we have not reached this point yet. Therefore, improving the training data should in turn improve model performance.

The question then becomes: \emph{which types of data should we use to predict protein function?} It will be extremely useful to understand which data are most predictive of protein function, separately or in combination, to guide experimental and curational data collection efforts going forward. To some extent, the answer depends on what the question is. On one hand, sequence data is ubiquitously available, so can be used to build general function prediction methods (\ref{chapter:metagenomes,chapter:yeast}). Network data, on the other hand, is powerful, but is essentially limited to model organisms, as network data is nonexistant for novel and neglected species (\ref{chapter:fly,chapter:network-fusion,chapter:yeast}). Targeted molecular biology and high-throughput screens are possible for culturable species (\ref{chapter:metagenomes}), but limited to smaller organisms with short life spans (\ref{chapter:yeast}).

In addition, it will be useful to understand which models, given the optimal training data, are most predictive of protein function. Neural networks have shown great promise in recent years and may prove to be the model of choice for protein function prediction. However, whilst neural networks are flexible models, their application to biological sequences is not yet as flexible as HMMs, which have performed well in previous CAFA challenges. Analysis of the best performing methods in the CAFA challenges will help to shed some light on which types of data and which models are most predictive of function. This will be especially true for CAFA 4, for which models would have had a great deal more training data available than for CAFA 3 and neural networks were a more popular choice of method.

\section{Predicted functions need experimental validation}

Predicting functions for proteins is easy; the challenge lies in predicting the correct functions. All predictive methods make trade offs, but on the whole methods wish to minimise the false positive and false negative rates (incorrect predictions) and maximise the true positive and true negative rates (correct predictions). Predictions must be validated by experimental observations that confirm whether the protein performs the predicted function. Experimental validation is useful for confirming true positives and refuting false positives, as these predictions will be contained in the set of predictions generated by a model. However, this strategy is not so useful at identifying true, and false, negatives because these instances may not be contained in the set of predictions. Furthermore, experimental validation can only be applied to functional labels that are actually predicted. There are on the order of \num{44000} GO terms, so models are usually trained to predict a subset of these terms. If a term is not predicted, it cannot be validated.

The work presented in this thesis predicted protein function under protein-centric (\ref{chapter:fly,chapter:network-fusion,chapter:yeast}) and function-centric (\ref{chapter:metagenomes,chapter:yeast}) models. These predictions will be validated by our experimental collaborators prior to publication of the work. Our predictions can also be used to guide targeted functional experiments in higher (model) organisms. For example, the proteins that we identified that are dysregulated in Alzheimer's disease in fly brains could be used to design experiments in mice.

Our predictions will help our collaborators to design the functional experiments and phenotypic screens that will be used to validate our predictions. Human intuition and experience will ultimately guide the experiments, according to availablity of resources and the ease with which particular functions can be validated. Jürg Bähler's group at UCL are in the process of validating a selection of our highest confidence fission yeast predictions. Once we have validated these predictions, we will submit our study for publication. In due course, once our group has developed our predictive pipeline for plastic hydrolase sequences, Florian Hollfelder's group at Cambridge will validate these sequences for their efficacy and efficiency in breaking down plastics.

\section{Expansion of FunFams via new methods and data}

In \ref{chapter:metagenomes}, we introduced FRAN, an algorithmic framework to generate FunFams on arbitrarily large numbers of sequences. This method will be crucial to capture information from the large volumes of diverse proteins that are being sequenced in conventional sequencing projects and metagenomics. Doing so will increase the quality of FunFams because the depth and diversity of FunFam alignments will increase. Better FunFams beget better function predictions, due to increased ability to identify specificity-determining positions that determine particular functions. The need for high-quality and high-coverage protein function predictions continues to grow. With reference to this thesis, better predicted functions could be applied to the yeast proteome (\ref{chapter:yeast}), to uncover other plastic hydrolases (\ref{chapter:metagenomes}), or to generate more accurate predictions for future CAFA competitions (\ref{chapter:yeast}).

Metagenomics generates unprecedented numbers of protein sequences, many of which are from novel species that live in underrepresented biomes. We would like to capture this information in CATH, Gene3D and FunFams. However, our current methods cannot scale to such behemothic data sets. In the future, our group will use FRAN to generate FunFams using Gene3D hits from UniProt and MGnify. Doing so will help to improve FunFams and protein functions predicted using them. This is an exciting new direction for CATH, which will help the database and methods to remain competitive when faced with an onslaught of competition from neural network-based methods.

We will use the findings from the analyses performed in the plastic hydrolase project to improve FunFHMMer.
The FunFHMMer algorithm was developed, tuned and benchmarked using only three of the superfamiles in CATH \cite{Das2015b}.
FunFams were then generated for all superfamilies in CATH.
Whilst FunFHMMer works well on the three superfamiles used to develop FunFHMMer, and produces high-quality FunFams for these superfamilies, we know that FunFHMMer does not generate such high-quality FunFams for other superfamilies.
For example, during our search for novel plastic hydrolases, our analysis of the α/β hydrolase superfamily FunFams has demonstrated that FunFHMMer may be oversplitting sequences into too many FunFams.
Compared with CATH v4.2, the latest version, v4.3, has many more FunFams for the α/β hydrolase superfamily.
One reason for this may be that v4.3 contains more sequences that are more diverse, so, in turn, these sequences will segregate into more families, each with different SDPs and, therefore, functions.
However, we have recently noticed that FunFam alignments tend to have low sequence diversity, as measured by the Neff score for the number of effective sequences in an alignment \cite{Altschul1997,Peng2010}.
In general, sequence diversity in alignments is good for structure prediction, but not for function prediction.
As we have recently begun a collaboration that uses FunFams for structure prediction, we are exploring ways to merge FunFams to create `StructFams' of more diverse sequences that are better for structure prediction.
We hope that these improvements will produce better FunFams for all superfamilies.

\section{Broaden the search for plastic hydrolases}

In \ref{chapter:metagenomes}, we identified putative PET hydrolases in metagenomes. $6\%$ of microbiome samples in MGnify have, so far, been assembled---just the tip of the iceberg---leaving a mountain of information to be mined. The next stage of this project will be to perform targeted assembly of samples from particular biomes to generate more metagenomic protein sequences (\ref{fig:biome_funfam_sequence_counts_scatter}). We will choose samples from biomes that look promising for finding plastic-degrading enzymes, whether that be from biomes that:
\begin{itemize}
    \item contain more ABH domains than expected by chance,
    \item contain proteins with high sequence identity to PETase, or
    \item from manual examination of biomes by curators.
\end{itemize}
In \ref{fig:biome_funfam_sequence_counts_scatter}, we only plotted high-level biomes that do not convey very specific information about the environments that samples were collected from (\ref{sec:mgnify-methods}). But when selecting samples to be assembled, we will consider more specific biome assignments using the GOLD biome ontology (\ref{sec:mgnify-methods}) \cite{Mukherjee2019}. We will predict proteins from the assembled contigs, which will be analysed for their similarity to PETase and their plastic-degrading potential.

We hope that this iterative pipeline will produce a wealth of information that can be analysed to discover new plastic hydrolases in nature. Putative sequences will be functionally validated using picodroplet functional metagenomics \cite{Colin2015} in a collaboration with Florian Hollfelder at Cambridge. We will be able to validate between \numrange{10}{100} sequences using this method. Alongside true positives for positive controls, we will introduce mutations into PETase at key sites, identified using CATH and structural bioinformatics. These mutations may change the efficiency of PET degradation, or even change the function or substrate-specificity to another plastic. Furthermore, we will test putative sequences that we discover through analyses similar to those performed in this work. Our metagenomic search pipeline is not limited to PETases, but is a flexible method to search for proteins that carry out any arbitrary function. The only real requirement is that the function must be able to be validated experimentally to confirm whether sequences do have the predicted functions. We may also explore other enzymatic functions with the Hollfelder group.